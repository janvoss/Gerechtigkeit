\documentclass[border=10pt]{standalone}
\usepackage{smartdiagram} %für tikz
\usepackage[ngerman]{babel} %Sprache deutsch (für Silbentrennung)
\usepackage[utf8]{inputenc} %UTF-Code für Umlaute
\usepackage[T1]{fontenc} %Trennung von Wörtern mit Umlauten
\usepackage{lmodern} %Schriftart
%\usepackage{libertine}
\renewcommand*\familydefault{\sfdefault} %Serifenlose Schrift als Standard
\usepackage{microtype}
\usesmartdiagramlibrary{additions}
\usetikzlibrary{fit}
\usetikzlibrary{decorations.pathreplacing}

\tikzstyle{container} = [draw, rectangle, semithick, inner sep=0.3cm
]

%Aufzählungsstriche
\AtBeginDocument{
	\def\labelitemi{\normalfont\bfseries{--}}
}


\begin{document}

		\begin{tikzpicture}[
		every node/.style = {shape=rectangle, % is not necessary, default node's shape is rectangle
			rounded corners,
		%	draw, semithick,
			text width=5cm,
			align=center,
			node distance=0.1cm
		}
		]
		
%Überschriften
	\node (Beobachten)[text depth=.25ex  % Schrift immer auf der gleichen Höhe
   		]{\textbf{Beobachten}
     };
    
    \node (Beschreiben)[text depth=.25ex, right = of Beobachten
    ]{\textbf{Beschreiben}
    };

	\node (Erklären)[text depth=.25ex, right = of Beschreiben
	]{\textbf{Erklären}
	};

	\node (Bewerten)[text depth=.25ex, right = of Erklären
	]{\textbf{Bewerten}
	};
    
    \node (Verändern)[text depth=.25ex, right = of Bewerten
    ]{\textbf{Verändern}
    };

%Spalten
	\node[below= of Beobachten, align=left](){
	\begin{itemize}
		\item Auswahl der Sachverhalte
		\item Messung
		\item Dokumentation
	\end{itemize}
	};

\node[below= of Beschreiben, align=left](){
	\begin{itemize}
		\item Zusammenstellen der Beobachtungen
		\item Übersicht
		\item Zusammenfassung/""Systematisierung
	\end{itemize}
};

	\node[below= of Erklären, align=left](){
		\begin{itemize}
			\item Systematische Zusammenhänge
			\item \glqq Ursache\grqq, \glqq Wirkung\grqq
			\item Theorien und Modelle
			\item \dots
		\end{itemize}
	};

	\node[below= of Bewerten, align=left](){
		\begin{itemize}
			\item Normative Aussagen
			\item Vergleich von \glqq Ist\grqq\ mit \glqq Soll\grqq
			\item Bewertung abhängig von gewählten Erklärungen
		\item \dots
		\end{itemize}
	};

	\node[below= of Verändern](){
		\begin{itemize}
			\item Umsetzen in Gesetze, Vorschriften usw.
			\item Berücksichtgung des rechtlichen Rahmens
			\item \glqq Machbarkeit\grqq
			\item \dots
		\end{itemize}
	};
		
     
%Horizontale Linie unter Überschriften	
	\draw [
	transform canvas={yshift=-0.1cm}
	] (Beobachten.south west) -- (Verändern.south east);
	
%gestrichelte Linie zwischen Bewerten und Verändern
	\path [draw = none] (Bewerten) -- (Verändern) node [midway, yshift = .5cm](Mitte_Praxis){};
	\node[below= 5.5cm of Mitte_Praxis](Mitte_Praxis2){};
	\draw [dashed] (Mitte_Praxis) -- (Mitte_Praxis2);
	
% Zusammenfassung Analyse und Praxis

	\node[below= 4cm of Beobachten, text width=0cm](Analyse-l){};
	\node[below= 4cm of Bewerten,   text width=0cm](Analyse-r){};
	

	\draw [decorate,decoration={brace,amplitude=7pt, mirror}] (Analyse-l) -- (Analyse-r) node[midway, yshift= -.7cm]{\textbf{Analyse}};
	
	\node[below= 4cm of Verändern.south west, text width=0cm, xshift= .7cm](Verändern-l){};
	\node[below= 4cm of Verändern.south east, text width=0cm, xshift = -.7cm](Verändern-r){};
	\draw [decorate,decoration={brace,amplitude=7pt, mirror}] (Verändern-l) -- (Verändern-r) node[midway, yshift= -.7cm]{\textbf{Praxis}};
	
%Überschrift
	\path [draw=none] (Beobachten) -- (Verändern) node [midway] (Mittelpunkt) {};	
	\node[above= 1cm of Mittelpunkt, text width=15cm]{\huge Schritte der ökonomischen Analyse und Praxis};
	
	\end{tikzpicture}

\end{document}