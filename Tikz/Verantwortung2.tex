\documentclass[border=10pt]{standalone}
\usepackage{smartdiagram} %für tikz
\usepackage[ngerman]{babel} %Sprache deutsch (für Silbentrennung)
\usepackage[utf8]{inputenc} %UTF-Code für Umlaute
\usepackage[T1]{fontenc} %Trennung von Wörtern mit Umlauten
\usepackage{lmodern} %Schriftart
%\usepackage{libertine}
\renewcommand*\familydefault{\sfdefault} %Serifenlose Schrift als Standard
\usepackage{microtype}
\usesmartdiagramlibrary{additions}
\usetikzlibrary{fit}
\usetikzlibrary{decorations.pathreplacing}

\tikzstyle{container} = [draw, rectangle, semithick, inner sep=0.3cm
]

%Aufzählungsstriche
\AtBeginDocument{
	\def\labelitemi{\normalfont\bfseries{--}}
}


\begin{document}
	

		\begin{tikzpicture}[
		every node/.style = {shape=rectangle, % is not necessary, default node's shape is rectangle
			rounded corners,
			draw, semithick,
			text width=3cm,
			align=center,
			node distance=1.4cm
		}
		]

			\node(A){\textbf{Subjekt}%\\ 
		%	\begin{itemize}
		%		\item Einzelpersonen oder auch Kollektive?
		%		\item Anforderungen an Akteur
		%\end{itemize}
	};
			\node[right=of A] (X) {\textbf{Handlung}};
			\node[right=of X] (Folgen) {\textbf{Handlungsfolgen} mit Bezug auf \textbf{Verantwortungsobjekt}};	
			\path (A) -- (Folgen) node[midway, draw=none](Mitte){};
			\node [above=of Mitte](B) {\textbf{Instanz}};
			\node [right=of B](C) {\textbf{Adressat}};
			
			\node[below= of Mitte, text width=11cm,] (Hintergrund) {\textbf{normativer Hintergrund} (plus positive Vermutungen)};
			
			\draw[->, thick] (A) -- (X);
			\draw[->, thick] (X) -- (Folgen);
			
			\draw[<	->, thick, dashed] (A) -- (B);
			\draw[->, thick, dashed] (X) -- (B);
			\draw[->, thick, dashed] (Folgen) -- (C);
			\draw[<->, thick, dashed] (B) -- (C);
			
			\node [below= of Hintergrund, text width=11cm, draw=none, align=left] {Ein \textbf{Verantwortungssubjekt} ist vor einer \textbf{Verantwortungsin"|stanz} und gegenüber einem \textbf{Adressaten} der Verantwortung für die \textbf{Folgen} seiner Handlungen in Bezug auf ein \textbf{Verantwortungsobjekt} verantwortlich -- vor einem gegebenen \textbf{normativen Hintergrund }und der damit verbundenen positiven Vermutungen.};
	\end{tikzpicture}

\end{document}